% !TEX root = main.tex

% TODO:
% REMARK:
% . * Don't mention LHC, CMS, etc. in this section, just have a historical review ana provide physical motivation to the study

\section{Introduction}
\label{sec:Introduction}


	% alpha
	In the SM, the CP violation in the production of top quark-antiquark ($t\bar{t}$) pairs at the CERN LHC is predicted to be very small.
	
	LHC is a top quarks factory


	The violation of combination of charge conjugation(C) and parity(P) is called CP violation. The 

	% ttbar, taget, LHC 

	\subsection{CP Violation in $t\bar{t}$ and possible sources}
	\label{ssec:Intro_CPVpossible}

	The possible source, which is also the primary focus in this analysis, of the CPV in $t\bar{t}$ is the anomalous top quark coupling affecting both decay and production of $t\bar{t}$. The anomalous coupling in $t\bar{t}$ production might be induced by the $\textbf{chromo electric dipole moment(CEDM)}$ of the interaction with top quark. The original CEDM lagragian density is coupled with $\textbf{chromo magnectic dipole moment(CMDM)}$ defined as:

	% original L

	The $d_t^g$ and $a_t^g$ in Eq.*** are top CEDM and CMDM parameters; The $G^a_{\mu \nu}$ is the gluon field strength interacting with top quarks; the $\sigma^{\mu \nu}$=$(i/2) [\gamma^{\mu},\gamma^{nu}]$.
	The $d_t^g$ in the lagragian is the CP-odd CEDM parameter with both real part $Re(d_t^g)$ and imaginary part $Im(d_t^g)$. \cite{Zhou:1998wz} The real part of $d_t^g$ is referred to the dipersive part and the imaginary part of $d_t^g$ is also regarded as absorptive part. In the prooduction part, the top CEDM $d_t^g$ contributes to the CP violating amplitudes with the Feynman diagram depicted in Fig.***.:

	% qq->tt, gg->tt diagram

	After the Higgs boson's discovered and standard model's development, Eq.*** is not appropriate to the gauge invariance and symmetry of standard model. The form of lagrangian with gauge invariance is shown below:

	% adapted L after SM






	% mainly to detect CEDM, theory part of CEDM

	\subsection{Triple product observables to detect CP violation in LHC}
	\label{ssec:Intro_TPinLHC}

	% intro of naive T, T-odd, T-even. focus on T-odd
	% used 4 observables, why? corresponding to detector

	\subsection{Experimental and data analysis strategy}
	\label{ssec:Intro_ExpDAStr}

\FloatBarrier
