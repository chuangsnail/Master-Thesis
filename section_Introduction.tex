% !TEX root = main.tex

% TODO:
% REMARK:
% . * Don't mention LHC, CMS, etc. in this section, just have a historical review ana provide physical motivation to the study

\section{Introduction}
\label{sec:Introduction}


	% alpha
	In the SM, the CP violation in the production of top quark-antiquark ($t\bar{t}$) pairs at the CERN LHC is predicted to be very small.
	
	LHC is a top quarks factory


	The violation of combination of charge conjugation(C) and parity(P) is called CP violation. The 

	% ttbar, taget, LHC 

	\subsection{CP Violation in $t\bar{t}$ and possible sources}
	\label{ssec:Intro_CPVpossible}

	The possible source, which is also the primary focus in this analysis, of the CPV in $t\bar{t}$ is the anomalous top quark coupling affecting both decay and production of $t\bar{t}$. The anomalous coupling in $t\bar{t}$ production might be induced by the $\textbf{chromo electric dipole moment(CEDM)}$ of the interaction with top quark. The original CEDM lagragian density is coupled with $\textbf{chromo}$ $\textbf{magnectic}$ $\textbf{dipole}$ $\textbf{moment(CMDM)}$ defined as:

	% original L
	\begin{equation}
	\begin{split}
	L_{cdm} = \frac{g_s}{2} \bar{t} T^a\sigma^{\mu \nu}(a_t^g + i \gamma^5 d_t^g) t G^a_{\mu \nu}
	\label{eq:CEDM_L_ori}
	\end{split}
	\end{equation}
	\FloatBarrier

	The $d_t^g$ and $a_t^g$ in Eq.\ref{eq:CEDM_L_ori} are top CEDM and CMDM parameters; The $G^a_{\mu \nu}$ is the gluon field strength interacting with top quarks; the $\sigma^{\mu \nu}$=$(i/2) [\gamma^{\mu},\gamma^{nu}]$.
	The $d_t^g$ in the lagragian is the CP-odd CEDM parameter with both real part $Re(d_t^g)$ and imaginary part $Im(d_t^g)$. \cite{Zhou:1998wz} The real part of $d_t^g$ is referred to the dipersive part and the imaginary part of $d_t^g$ is also regarded as absorptive part. In the prooduction part, the top CEDM $d_t^g$ contributes to the CP violating amplitudes with the Feynman diagram depicted in Fig.\ref{Intro:fig:qqgg_tt}.

	% qq->tt, gg->tt diagram
	\begin{figure}[H]
	\centering
		\includegraphics[width=0.85\textwidth]{Figures/Intro/qqgg_tt.png}
	\caption{Tree level Feynman diagram for $qq \rightarrow t\bar{t}$ and $gg \rightarrow t\bar{t}$, the crosses in diagrams denote the vertices modified by top CEDM \cite{Zhou:1998wz}}
	\label{Intro:fig:qqgg_tt}
	\end{figure}
	\FloatBarrier

	After the Higgs boson's discovered and standard model's development, Eq.\ref{eq:CEDM_L_ori} is not appropriate to the gauge invariance and symmetry of standard model. That is to say, any new physics must come in the form of an effective Lagrangian that respects all the symmetries of the SM, in particular the SU(2)(weak interaction) symmetry. This symmetry forbids couplings in Eq.\ref{eq:CEDM_L_ori}. The form of lagrangian in Eq.\ref{eq:CEDM_L_ori} could be reserve but to adapt it, the correct gauge invariant, operators must include a Higgs field. The adaptive form is shown in Eq.\ref{eq:CEDM_L_ada}.

	% adapted L after SM
	\begin{equation}
	\begin{split}
	L_{cdm} = g_s \frac{d_{tG}}{\Lambda^2} \bar{q_3} T^a\sigma^{\mu \nu} t \widetilde{\phi} G^a_{\mu \nu} + H.c.
	\label{eq:CEDM_L_ada}
	\end{split}
	\end{equation}
	\FloatBarrier

	where the $q_3$ is the third generation SM quark doublet, $\widetilde{\phi}$ is the scalar doublet, and the $T^a$ are the SU(3) generators. The relation between $d_{tG}$ and $d_t^g$ is shown(Eq.\ref{eq:CEDM_L_connection}). then the absorptive part of form factor $d_t^g$ which arises from a loop should not be written down in lagrangian.

	\begin{equation}
	\begin{split}
	d_t^g = \frac{\sqrt{2} v}{\Lambda^2} Im(d_{tG})
	\label{eq:CEDM_L_connection}
	\end{split}
	\end{equation}
	\FloatBarrier

	In addition to the CPV from CEDM in production of $t\bar{t}$, there are also CP violation, which is the anomalous couplinng $f$, arise in the vertices of $t\bar{t}$ with the decay $t \rightarrow bW^+$ and $\bar{t} \rightarrow \bar{b}W^-$, and would arise in the vertices of $t\bar{t}$. The $f$ is composed of CP conserving phase $\delta_f$ and CP violating phase $\phi_{f}$ with $\widetilde{f}$ parametrised. The vertices are shown below Eq.\ref{eq:CEDM_CPV_decay}, ref.\cite{PhysRevD.81.034013}:

	\begin{equation}
	\begin{split}
	\Gamma^{\mu}_{W_{tb}} = - \frac{g}{\sqrt{2}}V_{tb}^{*} \bar{u}(p_{b})[ \gamma_{\mu} P_{L} - i\widetilde{f}e^{i(\phi_f + \delta_f)} \sigma^{\mu \nu} (p_t - p_b)_{\nu} P_R ] u(p_t) \\
	\bar{\Gamma}^{\mu}_{W_{tb}} = - \frac{g}{\sqrt{2}}V_{tb} \bar{v}(p_{\bar{t}})[ \gamma_{\mu} P_{L} - i\widetilde{f}e^{i(-\phi_f + \delta_f)} \sigma^{\mu \nu} (p_{\bar{t}} - p_{\bar{b}})_{\nu} P_L ] v(p_{\bar{b}})
	\end{split}
	\label{eq:CEDM_CPV_decay}
	\end{equation}

	These two CP violating anomalous couplings, $d_t^g$ and $\widetilde{f} \sin{\phi_f}$, have induced the T-odd correlation, so it is considered to using T-odd observables to detecting.(ref.\cite{PhysRevD.81.034013},\cite{PhysRevD.79.013013},\cite{PhysRevD.80.034013}) Besides, there are some probable models could be the source of CP violation in $t\bar{t}$ like 2HDM model. There are some discussion in section. The emphasis of this analysis is to focus on approach to the top CEDM phenomenon.

	% mainly to detect CEDM, theory part of CEDM

	\subsection{Triple product observables to detect CP violation in LHC}
	\label{ssec:Intro_TPinLHC}

	% intro of naive T, T-odd, T-even. focus on T-odd
	% used 4 observables, why? corresponding to detector

	To approach the possibility of CP violation in sample, the time reversal would be introduced because of the assumption of CPT conservation which indicates that all the physical states are the eigenstates of CPT operator in the universe. Therrefore, the observables we used in the analysis are the triple product observables which are exactly $\textbf{T-odd observables}$. The $T$ here means the time reversal operator, more accurately, there is the $\textbf{naive time reversal operator}$ $\textbf{T}_N$. The ordinary time reversal operator means that it changes signs of the momenta and spins of all particles and also exchange the initial and final states. The naive time operator, however, is the time reversal without interchanging between initial state and final state because the initial state's detection and calculations are not available for usual decay process in experiment. That is to say, if the combination of charge and parity is not conserved, the kinematics of final state would tell the CP-violation story. The applied observable -- T-odd triple product observables, which basic form is shown below(Eq.\ref{eq:triple_product_form}):(ref.\cite{PhysRevLett.58.451},\cite{PhysRevD.80.034013}, )

	\begin{equation}
	O = \vec{v_{1}} \cdot ( \vec{v_2} \times \vec{v_{3}} )
	\label{eq:triple_product_form}
	\end{equation}

	where the $v_{1,2,3}$ are the momentum or spins vectors. It is obvious to be T-odd($T_{N}$-odd) type because of there are three components. Furthermore, there are also some T-even observables but not the usage of this analysis. Also, there are more combinations about these three vector, so CP-odd and CP-even type of triple product might have appeared. The CP-odd(CP-even) means the eigenvalue of CP eigenstate is -1(+1). The prefered triple product observables are commonly the CP-odd observables, take the observable $O = \vec{p_{l^-}} \cdot ( \vec{p_{t}} \times \vec{p_{\bar{t}}} )$ in \cite{PhysRevLett.58.451} for example:

	\begin{equation}
	\begin{split}
	p_{t}\xrightarrow[\text{}]{\text{CP}} -p_{\bar{t}} \; \; , \; \; p_{t}\xrightarrow[\text{}]{\text{CP}} -p_{\bar{t}} \; , \; \; \\
	p_{l^-} \xrightarrow[\text{}]{\text{CP}} p_{l^-} \; \; , \; \; p_{l^-} \xrightarrow[\text{}]{\text{CP}} p_{l^+} \;, \\
	\vec{p_{l^-}} \cdot ( \vec{p_{t}} \times \vec{p_{\overline{t}}} ) \xrightarrow[]{\text{CP}} \vec{p_{l^-}} \cdot ( - \vec{p_{\overline{t}}} \times - \vec{p_{t}} ) \\
	= \vec{p_{l^-}} \cdot ( \vec{p_{t}} \times \vec{p_{\overline{t}}} ) \;\;\;\;\;\;
	\end{split}
	\label{eq:ex_obs_eett}
	\end{equation}

	where the observable is CP-odd. We only put emphasis on the CP-odd observables. Given a CP-odd observable, we can define an asymmetry measurement in Eq.\ref{eq:asymmetry_form} which is established on counting events between positive and negative value of observable. The $O_i$ in Eq.\ref{eq:asymmetry_form} means the CP-odd observables and different observables have different sensitivity to the CP violation if there is new physics involved.

	\begin{equation}
	A_{cp} = \frac{ N_{events}(O_i>0) - N_{events}(O_i<0) }{ N_{events}(O_i>0) + N_{events}(O_i<0) }
	\label{eq:asymmetry_form}
	\end{equation}

	% development of triple product observable
	Under the form Eq.\ref{eq:triple_product_form} and the CP-odd design Eq.\ref{eq:asymmetry_form}, we can easily build T-odd observables for detecting CP violation. \\
	
	To discuss about the general case of T-odd triple product observable, there is a simple case -- the $\textbf{inclusive case}$, which is the jets' information used only. The parity operator would change the sign of four momentum, but the charge conjugate could not change the jets' sign because it is summed over by particles and antiparticles in defining the jets. There are 3 jets' momenta ordered by energy or spread by $p_{\perp}$($E_1$>$E_2$>$E_3$ or $p_{\perp 1}$ > $p_{\perp 2}$ > $p_{\perp 3}$). They are also used to form a CP-odd triple product $J$(Eq.\ref{eq:inclusive_obs1}, ref.\cite{PhysRevLett.58.451}):

	\begin{equation}
	J = \vec{p}_1 \cdot \vec{p}_2 \times \vec{p}_3
	\label{eq:inclusive_obs1}
	\end{equation}

	however, there is a adapted one with beam direction $\vec{p}$ and some total polarization $\vec{S}$ perpendicular to the beam direction. They both are unchange under CP operation, so the rest one is dropped a jet four momentum to make it CP-odd still. The form is shown in Eq.\ref{eq:inclusive_obs2}(ref.\cite{PhysRevLett.58.451}.

	\begin{equation}
	\widetilde{J} = \vec{p}_- \times \vec{S} \cdot \vec{p}_{jet}
	\label{eq:inclusive_obs2}
	\end{equation}

	Another case is the semi-inclusive process. When the particles identification is improved, the particles is not merely labeled by its four momentum. Therefore, there are wider tests could be adopted to do CP-test, and it is exactly more effective in the modern collider experiments. We denote the flavor of particles as $F$ and their anti-particles as $\bar{F}$, the possible observable terms are shown(Eq.\ref{eq:semi-inclusive_obs}ref.\cite{PhysRevLett.58.451}):

	\begin{equation}
	\begin{split}
	\vec{p}_- \times \vec{p}_{jet} \cdot ( \vec{p}_F - \vec{p}_{\bar{F}} ) \;\;, \;\; \vec{S} \times \vec{p}_{jet} \cdot ( \vec{p}_F - \vec{p}_{\bar{F}} ) \\
	\vec{p}_- \times \vec{S} \cdot ( \vec{p}_F + \vec{p}_{\bar{F}} ) \;\;, \;\; \vec{p}_{jet,1} \times \vec{p}_{jet,2} \cdot ( \vec{p}_F + \vec{p}_{\bar{F}} ) \\
	\vec{p}_- \cdot ( \vec{p}_F \times \vec{p}_{\bar{F}} ) \;\;,\;\; \vec{S} \cdot ( \vec{p}_F \times \vec{p}_{\bar{F}} )
	\label{eq:semi_inclusive_obs}
	\end{split}
	\end{equation}

	the $(\vec{p}_F + \vec{p}_{\bar{F}})$ means the particle and anti-particle are not necessary to be really distinguished, for example, it is acceptible that the charges of 2 same flavor jets are not clearly assigned under the type of observable. On the other hand, the $(\vec{p}_F - \vec{p}_{\bar{F}})$ and $(\vec{p}_F \times \vec{p}_{\bar{F}})$ need more accuracy at seperating particle and anti-particle. There is also a CP-odd type but T-even observables like Eq.\ref{eq:3vec_Teven}, but it is not used in this real data analysis. T-even observables will be discussed more in section.\ref{ssec:AcpObs}.

	\begin{equation}
	\vec{p}_{jet} \cdot ( \vec{p}_F - \vec{p}_{\bar{F}} )
	\label{eq:3vec_Teven}
	\end{equation}

	It is necessary to find out the suitable observables for this analysis. There are more organized T-odd triple product observables in ref.\cite{Hayreter:2015ryk}

	% related to this analysis

	The analysis is to probe CP violation in $t\bar{t}$ semileptonic decay Eq.\ref{eq:semileptonic_decay}, where both top quarks decay to bottom quarks and W bosons. Then one of W bosons decays and ends up with being recognized as 2 jets($j_1$,$j_2$), and the other one decays to lepton and neutrino. 

	\begin{equation}
	t\bar{t} \rightarrow b\bar{b} W^+W^- \rightarrow b\bar{b} jj l \nu
	\label{eq:semileptonic_decay}
	\end{equation}

	According to the final states' resolution and reconstruction, there are four chosen appropriate observables:

	\begin{equation}
	\begin{split}
	O_3 = Q_l \epsilon(p_{b}, p_{\bar{b}}, p_{l}, p_{j_1}) \xrightarrow[\text{}]{b \bar{b} CM} Q_l \vec{p_{b}} \cdot ( \vec{p_l} \times \vec{p_{j_1}} ) \\
	O_{6} = Q_l \epsilon(P, p_{b}-p_{\bar{b}}, p_{l}, p_{j_1}) \xrightarrow[\text{}]{lab} Q_l (\vec{p_{b}} - \vec{p_{\overline{b}}}) \cdot ( \vec{p_l} \times \vec{p_{j_1}} ) \\
	O_{12} = q \cdot (p_{b}-p_{\bar{b}}) \epsilon(P_{}, q, p_{b}, p_{\bar{b}}) \xrightarrow[]{lab} (\vec{p_{b}} - \vec{p_{\overline{b}}})_z ( \vec{p_b} \times \vec{p_{\overline{b}}} )_z \\
	O_{14} = \epsilon(P, p_{b}+p_{\bar{b}}, p_{l}, p_{j_1}) \xrightarrow[]{lab} (\vec{p_{b}} + \vec{p_{\overline{b}}}) \cdot ( \vec{p_l} \times \vec{p_{j_1}} )
	\end{split}
	\label{eq:four_obs}
	\end{equation}

	Some of the possibly useful triple products are not really considered, for example, ... %


	\subsection{Experimental and data analysis strategy}
	\label{ssec:Intro_ExpDAStr}

\FloatBarrier
