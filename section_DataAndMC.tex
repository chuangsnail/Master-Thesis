% !TEX root = main.tex

% TODO:
% REMARK:

\section{Data and Simulation samples}
\label{sec:DataAndMC}

	\subsection{Data sample}
	\label{ssec:DataAndMC_Data}

	% remember to mention golden json

	\subsection{Simulation sample}
	\label{ssec:DataAndMC_MC}

	\subsection{Correction on simulation sample}
	\label{ssec:DataAndMC_corMC}

		% pileup introduction is set on "vertex"
		\subsubsection{Pile-up Reweighing}
		\label{sssec:DataAndMC_PU}

			% https://twiki.cern.ch/twiki/bin/viewauth/CMS/PileupMCReweightingUtilities?fbclid=IwAR0SuZFQ5Um0IfZn1-CHXia6NPMYe2_7cz2OGXxhCYvNvfl_tTBke-w22l8

			As mention in section.\ref{ssec:PhysObj_pu}, pileup is the 
			Although Monte Carlo sample may roughly cover the pileup interactions, the final distribution is also sensitive to the subtle content of primary vertex. Furthermore, the distribution of reconstructed vertices can be differently affected by the offline event selection cut and some high level trigger between real data and Monte Carlo sample. To fix the descrepancy between data and simulation, 

		\subsubsection{Jet Energy correction, smearing and resolution}
		\label{sssec:DataAndMC_JE_CSR}

		\subsubsection{Efficiency Scale Factor}
		\label{sssec:DataAndMC_EffSF}

		\subsubsection{Weigh to Luminosity}
		\label{sssec:DataAndMC_lumi}




%%%--- TODO ---%%%
%%% table of the MC sample's Xsec and generated events number and generator -> weight to luminosity

\begin{center}
\begin{tabular}{ c c c c c }
\hline
Process sample & Cross Section (pb) & k-factor & Events Number & Generator \\ 
\hline
$t$$\bar{t}\rightarrow b \bar{b}jjl\nu$ & XXX & 1 & CCCCCC & AAAAAA \\
\hline
cell7 & cell8 & XXXXXX & cell7 & cell8 \\
\hline  
\end{tabular}
\end{center}


\FloatBarrier
