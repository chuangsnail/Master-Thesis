% !TEX root = main.tex

% TODO:
% REMARK:

\section{Dilution Effect}
\label{sec:Dilution}
	
	As mentioned previously, the measured $A'_{cp}$ is the results from physical objects detected from detector and reconstructed by numerous technical process, so the $A'_{cp}$ may not perfectly demonstrate the original expected $A_{cp}$ in parton level. There might be a dilution factor to depict the process of dilution effect linearly, and it is shown in Eq.\ref{eq:Dilution_Acp}. Also, the dilution effect $D_i$ is dependent on observables.

	\begin{equation}
	A_{cp}(O_i) = D_i A'_{cp}(O_i)
	\label{eq:Dilution_Acp}
	\end{equation}
	\FloatBarrier

	We can approach the dilution effect through simulation sample of our signal -- $t\bar{t}$ MC sample. We let $t\bar{t}$ MC sample passing the full selection and calculate the $A'_{cp}$ by detector-level's physical objects(jets, muon, electron, MET...), at the same time, the event's real $A_{cp}$ could be derived by the generator-level particle's information. By comparing the sign of $O_i$ in detector-level and $O_i$ in generator-level we can get the $\emph{same sign}$ and $\emph{opposite sign}$ of $O_i$ between both levels' measurements in an event. This is because the CP-odd observable use the observable $O_i$ positive and negative event number rather than $O_i$ value's average or other. The dilution factor is gotten from Eq.\ref{eq:Dilution_epsilon} by opposite sign rate $\epsilon_{opp}$ which is calculate by number of same sign events($N_{same}$) and number of opposite events($N_{opp}$) in $t\bar{t}$ MC sample:

	\begin{equation}
	D_i = 1 - 2 \epsilon_{opp} \; \; , \; \; \; \; \; \; \epsilon_{opp} = \frac{N_{opp}}{N_{same} + N_{opp}}
	\label{eq:Dilution_epsilon}
	\end{equation}
	\FloatBarrier

	\subsection{Dilution factor}
	\label{Dilution:dilution_factor}
		
		The derivation of the dilution factor is showing in this section. For a sample of data, there must be correctly tagged, mis-identified, and mis-tagged states when detector detecting the physical objects. The sum of these three states' efficiency whould be unity.

		\begin{equation}
		\epsilon_{cor} + \epsilon_{misi} + \epsilon_{mist} = 1
		\label{eq:Dilution_unity}
		\end{equation}
		\FloatBarrier

		By simulation data, we need to know the relationship between $A_{cp}$ and $A'_{cp}$, which are composed by number of real positive and negative observables' sign($N^{+}$,$N^{-}$) and number of detected observables' sign($N'^{+}$,$N'^{-}$):

		\begin{equation}
		A_{cp} = \frac{ N^{+} - N^{-} }{ N^{+} + N^{-} } \;\;, \;\;\;\; 
		A'_{cp} = \frac{ N'^{+} - N'^{-} }{ N'^{+} + N'^{-} }
		\label{eq:Dilution_001}
		\end{equation}
		\FloatBarrier
		

		The $N^+$ and $N^-$ includes only correctly tagged and mis-identified cases ($N^{+/-}_{cor}$,$N^{+/-}_{misi}$) because the interested number do not contain mis-tagged type. Mostover, the condition only occurs in original generator-level's number (of positive and negative observables) ,so:

		\begin{equation}
		N^+ + N^- = (\epsilon_{cor} + \epsilon_{misi})N = (1-\epsilon_{mist})N = (1-\epsilon_{mist})(N'^+ + N'^-)
		\label{eq:Dilution_002}
		\end{equation}
		\FloatBarrier

		where $N$ is the total number of events and it is the same as $N'^+ + N'^-$ but the interested number in generator level($N^+$, $N^-$) are just the part of $N$. Also, the number in detected observables would be seperated to 3 types which are correctly-tagged, mis-identified, and mis-tagged($N'^{+/-}_{cor}$, $N'^{+/-}_{misi}$, $N'^{+/-}_{mist}$). It is known that the mis-tagged type number in positive and negative may be fairly distributed. That is, 

		\begin{equation}
		N'^+_{mist} = N'^-_{mist} = \frac{1}{2}\epsilon_{mist} \cdot N
		\label{eq:Dilution_003}
		\end{equation}
		\FloatBarrier

		Therefore, by the Eq.\ref{eq:Dilution_002}, \ref{eq:Dilution_003}, 

		\begin{equation}
		\begin{split}
		N'^+ = N'^+_{cor} + N'^+_{misi} + N'^+_{mist} = (\frac{\epsilon_{cor}}{\epsilon_{misi} + \epsilon_{mist}})N^+ + (\frac{\epsilon_{misi}}{\epsilon_{cor} + \epsilon_{misi}})N^- + \frac{1}{2}\epsilon_{mist} N \\
		N'^- = N'^-_{cor} + N'^-_{misi} + N'^-_{mist} = (\frac{\epsilon_{cor}}{\epsilon_{misi} + \epsilon_{mist}})N^- + (\frac{\epsilon_{misi}}{\epsilon_{cor} + \epsilon_{misi}})N^+ + \frac{1}{2}\epsilon_{mist} N  \\
		\Rightarrow N'^+ - N'^- = (\frac{\epsilon_{cor}-\epsilon_{misi}}{\epsilon_{cor}+\epsilon_{misi}})(N^+ - N^-) \;\;\;\;\;\;\;\;\;\;\;\;\;\;\;\;\;\;\;\;\;\;\;\;\;\;\;\;\;\;\;\;\;\;\;\;\;\;\;\;\;\;\;\;\;\;\;\;\;\;\;\;\;\;\;\;\;\;\;\;\;\;
		\end{split}
		\label{eq:Dilution_004}
		\end{equation}
		\FloatBarrier

		And the relationship between $A'_{cp}$ and $A_{cp}$ is clear:

		\begin{equation}
		\begin{split}
		A'_{cp} = \frac{N'^+ - N'^-}{N'^+ + N'^-} \\
		= (\frac{1}{N'^+ + N'^-})[(\frac{\epsilon_{cor}-\epsilon_{misi}}{\epsilon_{cor}+\epsilon_{misi}})(N^+ - N^-)] \\
		= (\frac{1}{N^+ + N^-})(\frac{\epsilon_{cor}-\epsilon_{misi}}{\epsilon_{cor}+\epsilon_{misi}})(N^+ - N^-)\\
		= (\epsilon_{cor}-\epsilon_{misi})\frac{N^+ - N^-}{N^+ + N^-}\\
		= D \cdot A_{cp}
		\end{split}
		\label{eq:Dilution_005}
		\end{equation}
		\FloatBarrier

		There are shown that the linear condition about $A'_{cp}$ and $A_{cp}$. By using $t\bar{t}$ simulation sample, the dilution factor can be measured by correctly tagged and mis-identified type efficiency because of the given no mistag type from MC truth studying. The simple form by correctly and incorrectly tagged efficiencies is listed below(Eq.\ref{eq:Dilution_006}):

		\begin{equation}
		D = \epsilon_{cor} - \epsilon_{incor} = (1-\epsilon_{incor}) - \epsilon_{incor} = 1 - 2\epsilon_{incor}
		\label{eq:Dilution_006}
		\end{equation}
		\FloatBarrier

	\subsection{Result}
	\label{Dilution:result}
		
		The incorrectly tagged efficiency $\epsilon_{incor}$ in Eq.\ref{eq:Dilution_006} conform to the opposite efficiency $\epsilon_{opp}$ in Eq.\ref{eq:Dilution_epsilon}, so the results of dilution effects could be handed by calculating the simulation sample. The different signal reconstruction method and selection strategy also affect the final decision of choosing obsjects to be calculated. Then the effect may vary the performance on dilution effect, so there are dilution factors and same sign ratios of four observables of three event-reconstruction and selection method.(Table. \ref{Dilution:tb:chi2}, \ref{Dilution:tb:MVAA}, \ref{Dilution:tb:MVAB})

		\begin{center}
		\setlength{\tabcolsep}{12pt}
		\begin{longtable}{ c | c c }
		\caption{Same sign ratio and dilution factor ($\chi^2_{min}$ result)}\\
		Observable & same sign rate[\%]($=1-k$) & Dilution Factor[\%] \\
		\hline
		$O_{3}$ & 70.16$\pm$0.03  &  40.33$\pm$0.05  \\
		$O_{6}$ &  68.72$\pm$0.03  &  37.45$\pm$0.05  \\
		$O_{12}$ &  84.03$\pm$0.02  &  68.07$\pm$0.04  \\
		$O_{14}$ &  76.65$\pm$0.02  &  53.31$\pm$0.05  \\
		\hline
		\end{longtable}
		\label{Dilution:tb:chi2}
		\end{center}

		\begin{center}
		\setlength{\tabcolsep}{12pt}
		\begin{longtable}{ c | c c }
		\caption{Same sign ratio and dilution factor (MVA-A result)}\\
		Observable & same sign rate[\%]($=1-k$) & Dilution Factor[\%] \\
		\hline
		$O_{3}$ & 70.16$\pm$0.03  &  40.33$\pm$0.05  \\
		$O_{6}$ &  68.72$\pm$0.03  &  37.45$\pm$0.05  \\
		$O_{12}$ &  84.03$\pm$0.02  &  68.07$\pm$0.04  \\
		$O_{14}$ &  76.65$\pm$0.02  &  53.31$\pm$0.05  \\
		\hline
		\end{longtable}
		\label{Dilution:tb:MVAA}
		\end{center}

		\begin{center}
		\setlength{\tabcolsep}{12pt}
		\begin{longtable}{ c | c c }
		\caption{Same sign ratio and dilution factor (MVA-B result)} \\
		Observable & same sign rate[\%]($=1-k$) & Dilution Factor[\%] \\
		\hline
		$O_{3}$ & 70.16$\pm$0.03  &  40.33$\pm$0.05  \\
		$O_{6}$ &  68.72$\pm$0.03  &  37.45$\pm$0.05  \\
		$O_{12}$ &  84.03$\pm$0.02  &  68.07$\pm$0.04  \\
		$O_{14}$ &  76.65$\pm$0.02  &  53.31$\pm$0.05  \\
		\hline
		\end{longtable}
		\label{Dilution:tb:MVAB}
		\end{center}


\FloatBarrier
