
\thispagestyle{empty} %%% remove page number of cover
\begin{abstract}

The research is to measure the CP violation in top quark pair at $\sqrt{s}=13$TeV in CMS experiment. It is based on the $\sqrt{s}=8$TeV energy scale result published in JHEP. According to the data collected from pp collider with integrated luminosity $35.9fb^-1$ in 2016 run, the asymmetry will be measured by T-odd triple product observables with specific final states which are exact one lepton and at least four jets. The uncertainty, background interference, and the dilution effect of the analysis have been well prepared. Besides, the T-even observables and new physics sources of $t\bar{t}$ CPV are discussed.

\newpage

\begin{center}
    \fontsize{16pt}{27}\selectfont{摘要}
\end{center}

\fontsize{12pt}{21}\selectfont{
    \quad 建立在質心能量八兆電子伏特的分析結果之上,本論文試圖於緊湊緲子螺線圈實驗中,質心能量十三兆電子伏特下,測量頂夸克對的共軛宇稱破壞。從2016年質子對撞累積光度35.9 逆飛靶恩的數據中,我們可以藉由逆時間運算子擁有特徵值為-1的觀察量根據一個輕子及至少四個噴流的最終態來計算頂夸克對中的不對稱性。在這分析中,不確定性資訊,背景的干涉程度,以及偵測器的影響都已經被考慮並準備妥當。除此之外,其他可能的不對稱新物理來源以及逆時間運算子擁有特徵值為1的觀察量也同時在這篇論文中被討論。
}

\end{abstract}
\FloatBarrier
