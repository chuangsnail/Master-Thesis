% !TEX root = main.tex

% TODO:
%   * Acronym to be covered: \sqrt{s}, $pp$
%   * Make a reference to post-run-I appendix
% REMARK:
% . * LHC, CMS are both defined in this section

\clearpage
\section{Experimental Apparatus}
\label{sec:ExperimentalAppratus}
% http://www.lhc-closer.es/taking_a_closer_look_at_lhc/1.lhc_parameters
% https://home.cern/resources/faqs/facts-and-figures-about-lhc

	\subsection{Large Hadron Collider(LHC)}
	\label{ssec:ExpApp_LHC}

		The $\textbf{large}$ $\textbf{hadron}$ $\textbf{collider}$$\textbf{(LHC)}$ is the largest and highest-energy-scale particle collider in the world. The LHC is under European Organization for Nuclear Research(CERN) which is well known high energy physics organization center. Many of discoveries of high energy physics are confirmed by the organization. The LHC is 27km long located at the France-Switzerland border and 175m deep underground to reduce most cosmic ray which could affect the experimental instruments. The LHC is designed the collision energy $\sqrt{s} = 13TeV$. There are several stages for accelerating -- from linear accelerator (LINAC) with the starting speed in 0.3c, and then go throgh the proton synchrotron(PS) to 0.87c, and accelerating in super proton synchrotron(SPS) and LHC to 0.999999c(The illustration is shown in Fig.\ref{ExpApp:fig:LHC_chain}). 

		\begin{figure}[H]
		\centering{}
	    	\includegraphics[width=0.8\textwidth]{Figures/ExpApparatus/LHC_chain2.png}\\
		\caption{LHC chain illustrration plot\cite{Halkiadakis:2010mj}}
		\label{ExpApp:fig:LHC_chain}
		\end{figure}
		\FloatBarrier

		There are four largest experiments on LHC -- A Toroidal LHC ApparatuS(ATLAS), Compact Muon Solenoid(CMS), LHCb, A Large Ion Collider Experiment(ALICE). They are designed to capture particles from collisions. ATLAS and CMS are designed for almost the same purpose which aims at the highly massive particles because of the unprecedented energy scale and at searching for the evidence of theories beyond Standard Model(BSM). The LHCb is an asymmetric detector which is designed mainly for the B(b-quark mesons) physics. ALICE is an heavy ion experiment which is interested in the collision of proton-lead and lead-lead. The logos of these four experiments are listed below(Fig.\ref{ExpApp:fig:logos}).
		
		\begin{figure}[H]
		\centering
		    \subfigure{\includegraphics[width=0.25\textwidth]{Figures/ExpApparatus/CMS_logo.png}}
		    \subfigure{\includegraphics[width=0.45\textwidth]{Figures/ExpApparatus/ATLAS_logo.png}}\\
		\end{figure}
		\FloatBarrier
		\begin{figure}[H]
		\centering
		    \subfigure{\includegraphics[width=0.25\textwidth]{Figures/ExpApparatus/ALICE_logo.png}}
			\subfigure{\includegraphics[width=0.35\textwidth]{Figures/ExpApparatus/LHCb_logo.jpg}}\\
		\caption{The logos of CMS, ATLAS, ALICE and LHCb experiments}
		\label{ExpApp:fig:logos}
		\end{figure}
		\FloatBarrier

		


	\subsection{Compact Muon Solenoid(CMS) Detector}
	\label{ssec:ExpApp_CMS}

		The structure of CMS detector is a large solenoid. It is 13 meters long with 6 meters inner diameter. There is also a superconducting solenoid in with 3.8T magnetic field which provides a large bending power(12Tm) before the measurement of muon's bending angle. There is also the iron yoke supporting the coil and returning the magnetic flux. Inner side of superconducting solenoid are pixel and strip tracker, and going outward with $\textbf{electromagnetic}$ $\textbf{calorimeter}$$\textbf{(ECAL)}$ and $\textbf{hadron}$ $\textbf{calorimeter}$$\textbf{(HCAL)}$. Then the one outside the superconducting solenoid is the muon detector. The structure diagram is shown in Fig.\ref{ExpApp:fig:CMS_structure} .The detail could be checked in ref.\cite{Chatrchyan:2008aa}

		\begin{figure}[H]
		\centering{}
	    	\includegraphics[width=0.8\textwidth]{Figures/ExpApparatus/CMS_detector.png}\\
		\caption{CMS structure\cite{Chatrchyan:2008aa}}
		\label{ExpApp:fig:CMS_structure}
		\end{figure}
		\FloatBarrier

		At the same time, the detector's slices of different functions are shown in Fig.\ref{ExpApp:fig:CMS_slices}(ref.\cite{Barney:2120661})

		\begin{figure}[H]
		\centering{}
	    	\includegraphics[width=0.8\textwidth]{Figures/ExpApparatus/CMSslice_whiteBackground.png}\\
		\caption{CMS slices\cite{Barney:2120661}}
		\label{ExpApp:fig:CMS_slices}
		\end{figure}
		\FloatBarrier

		The CMS use the right-handed coordinate, that is, the x-axis points to the center of LHC, y-axis points up and orthogonal to the ground, and z-axis is along with anti-clockwise beam direction. The angle $\phi$ is the azimuthal angle on the x-y plane;The other angle $\theta$ means the polar angle which is the angle between the positive z-axis and the observed point. Instead, in LHC the pseudo-rapidity $\eta = -\ln{\tan{(\theta/2)}}$ is more common used because $\eta$ is lorentz invariance under frame transformation. The other usuall usage is the $"$transverse$"$, for example, transvers momentum $p_T$ means the projection of momentum on the x-y plane which is exactly $p_T = \sqrt{p_x^2 + p_y^2}$.

		The different physical objects would be caught by different part of CMS detector. The muon has bended trajectory in tracker and muon chambers but pass without interacting with ECAL; An electron is seen from combined information from tracker(cureved trajectary) and ECAL; A photon is fully deposited on ECAL without reaction passing tracker; The hadrons also have no interaction with track system, but the charged hadron's hadronization and showering could be approached with both ECAL and HCAL.

		\subsubsection{Tracking system}
		\label{sssec:ExpApp_tracking}

			The CMS tracker's schematic drawing is shown below in Fig.\ref{ExpApp:fig:tracker} from reference\cite{Chatrchyan:2014fea} where $\textbf{strip tracker}$ with small silicon tracker within. 

			\begin{figure}[H]
			\centering{}
		    	\includegraphics[width=0.85\textwidth]{Figures/ExpApparatus/tracker.png}\\
			\caption{The drawing of the CMS inner tracking system\cite{Chatrchyan:2014fea}}
			\label{ExpApp:fig:tracker}
			\end{figure}
			\FloatBarrier

			Futhermore, the tracker is immersed in magnetic field from CMS solenoid. There are four subsystem in it and they are completed by endcaps(|pseudorapidity $\eta | < 2.5$) on either sides of barrels. The four subsystems are the $\textbf{Tracker}$ $\textbf{Inner}$ $\textbf{Barrel}$$\textbf{(TIB)}$, the $\textbf{Tracker}$ $\textbf{Inner}$ $\textbf{Disk}$$\textbf{(TID)}$, the $\textbf{Tracker}$ $\textbf{Outer}$ $\textbf{Barrel}$$\textbf{(TOB)}$, and the $\textbf{Tracker}$ $\textbf{EndCaps}$$\textbf{(TEC)}$. There are slices of silicons in these tracking subsystems, and the detail of thickness, position and layers would be found in reference\cite{Chatrchyan:2014fea}. Also, the summary of the principal characteristics of the various tracker subsystems from \cite{Chatrchyan:2014fea} is attached below\ref{ExpApp:fig:tracker_sum}:
			
			\begin{figure}[H]
			\centering{}
		    	\includegraphics[width=0.85\textwidth]{Figures/ExpApparatus/summary_subtracker.png}\\
			\caption{Summary of the principal characteristics of the various tracker subsystems\cite{Chatrchyan:2014fea}}
			\label{ExpApp:fig:tracker_sum}
			\end{figure}
			\FloatBarrier

		\subsubsection{Electromagnetic Calorimeter(ECAL)}
		\label{sssec:ecal}

			$\textbf{Electromagnetic Calorimeter(ECAL)}$ is a fine-grain, compact, homogeneous calorimeter which is composed of $PbWO_4$ scintillating crystal. Because of its small radiation length, radiation tolerance, and small Molie`re radius, the material is chose. The numerous crystals are arranged in $\textbf{central}$ $\textbf{barrel}$ $\textbf{section}$$\textbf{(EB)}$ and two $\textbf{endcaps}$$\textbf{(EE)}$ respectively. They line in cetral barrel section with the range $|\eta|<1.48$ as quasi-projective pattern. Also, the crystals line in endcaps covering the extensive range up to $|\eta|=3.0$. The transverse size of crystal is appropriate to the shower size in $PbWO_4$, being advantage for photon's shower-shape-based identification. However, $PbWO_4$ has low light yields, so there must be some photodetectors with internal amplification used to get available light yield signal. There are also the $\textbf{preshower}$ $\textbf{detector}$$\textbf{(ES)}$ which are 2 layers of lead absorber. The preshower detector is placed at the range $1.65<|\eta|<2.5$ and in the front side of the endcaps(EE), and the lead avsorbers are installed with orthogonal layers of silicon strip sensors. The main purpose of the preshowerr detector is to help separating $\pi_0$ and $\gamma$. The ECAL structure schemas are shown in Fig.\ref{ExpApp:fig:ECAL1} and Fig.\ref{ExpApp:fig:ECAL2} . The details could be checked in ref.\cite{ECAL_ex}.


			\begin{figure}[H]
			\centering{}
		    	\includegraphics[width=0.85\textwidth]{Figures/ExpApparatus/ECAL.png}\\
			\caption{ECAL structure schema-1\cite{ECAL_ex}}
			\label{ExpApp:fig:ECAL1}
			\end{figure}
			\FloatBarrier

			\begin{figure}[H]
			\centering{}
		    	\includegraphics[width=0.85\textwidth]{Figures/ExpApparatus/ECAL2.png}\\
			\caption{ECAL structure schema-2\cite{ECAL_ex}}
			\label{ExpApp:fig:ECAL2}
			\end{figure}
			\FloatBarrier

		\subsubsection{Hadronic Calorimeter(HCAL)}
		\label{sssec:hcal}

			$\textbf{Hadronic}$ $\textbf{Calorimeter}$$\textbf{(HCAL)}$ contains four subsystems -- the $\textbf{barrel}$$\textbf{(HB)}$, $\textbf{endcap}$$\textbf{(HE)}$, $\textbf{outer(HO)}$, and $\textbf{forward}$$\textbf{(HF)}$ calorimeters. The arrangement of them are checked in Fig.\ref{ExpApp:fig:HCAL1}. The HB and HE are inside the superconductor solenoid, and both of them have absorber material brass and scintillator which is active material. The HB which consists 2 half barrels covers the range $|\eta| < 1.3$ and the HE covers $1.305 < |\eta| < 3.0$. The HB is built of 18 wedges which cover 20 degrees in $\phi$ repectively, HE is manufactured by brass disks with scintillator wedges which also cover 20 degrees in $\phi$. They have same segmentation size except for near $|\eta| = 3.0$ where size of segmentation is doubled. For the HF, two different length quarts fibers embedded in the steel constitute it. The HF's photdetectors are eight-stage photomultipliers(PMT) to which the Cherenkov light generated from the fibers is transmitted in. There is the ensurement of eduquate sampling depth for central shower in the region $|\eta| < 1.26$ -- HO calorimeter. HO is located outside the superconductor solenoid. HO is used to be an additional absorber, to identify some late starting showers and to measure the deposits of showers' energy after going through HB. The previous content and more details of HCAL could b e found in ref.\cite{Chatrchyan:2008aa},\cite{Canko_ak_2009} and \cite{collaboration_2012}.

			\begin{figure}[H]
			\centering{}
		    	\includegraphics[width=0.85\textwidth]{Figures/ExpApparatus/HCAL1.png}\\
			\caption{HCAL structure schema\cite{Chatrchyan:2008aa}}
			\label{ExpApp:fig:HCAL1}
			\end{figure}
			\FloatBarrier

		\subsubsection{Muon Chambers}
		\label{sssec:muon_detector}

			The highlighting feature of CMS detector is the muon detector. The muon system in CMS is designed to have capacity of reconstructing the momentum and charge of muon completely under the available range of LHC's energy scale. The silicon tracker inside could measure the charge particles' momentum in the range $|\eta|<2.5$. In addition, the muon detector is located outside covering range $|\eta|<2.4$. By collecting positions on the multi-layers of stations and tracker record, the detectors trace the muons' path. The momentum record is based on the concept that the more momentum carried by muon, the less the bending angle of it. For part of hardware, there are 3 types of gaseous detector identifying muons. Those are gas ionization chambers which is appropriate to constitute CMS muon system: $\textbf{drift}$ $\textbf{tube}$ $\textbf{chambers}$$\textbf{(DTs)}$, $\textbf{cathode}$ $\textbf{strip}$ $\textbf{chambers}$$\textbf{(CSCs)}$, and $\textbf{resistive}$ $\textbf{plate}$ $\textbf{chambers}$$\textbf{(RPCs)}$. There are shown in structure diagram in Fig.\ref{ExpApp:fig:muon_chamber1}, the DTs are labeled MB($"$Muon Barrel$"$), the CSCs are labeled ME($"$Muon Endcap$"$), and the RPCs are inlayed in both the barrel and endcaps of CMS detector, so they are labeled RB and RE.

			\begin{figure}[H]
			\centering{}
		    	\includegraphics[width=0.85\textwidth]{Figures/ExpApparatus/muon_chamber.png}\\
			\caption{Muon detectors structure schema\cite{Sirunyan:2018fp}}
			\label{ExpApp:fig:muon_chamber1}
			\end{figure}
			\FloatBarrier

			The DT chambers cover the $\eta < 1.2$ rergion and are organized into four stations. The front three stations contain 8 chambers with 2 set of four chambers. They measured both the muon cooordinate in $\phi - r$ plane(which is equal to x-y plane) and z direction along the beam line. The forth station does the job only for the best angular resolution rather than on z-direction. To eliminate the dead spots in the efficiency, the chamers' drift cells are offset a half-cell width related to their neighbor; In the 2 regions of endcaps, the CSC which is characterized fast response time, fine segmentation, and radiation resistance can deal with high background level and large non-uniform magnetic field at $0.9 < |\eta| < 2.4$. Also with 4 stations perpendicular to beam line, the CSCs arrange radially outward and do the measurement precisely in the x-y plane. Besides, 6 layers of CSC robustly rejects non-muon background and provides efficient matching of hits to other stations and hits in tracker. Those are some of the reason why CMS has great muon-seen performance; The RPCs, which are double-gap chambers, have properties which are fast, independent, and having highly-segmented trigger. The trigger is a steep $p_T$ threshold at the portion of the range $|\eta| < 1.6$ of the muon system. In order to ensure good operating at high rates, avalanche mode would be operated by the chambers. The first 2 layers of 6 layers RPCs in barrel region are designed redundancy to give a low-$p_T$-tracks trigger algorithm which might have stopped before reaching outer stations. Ather structure plot is shown in Fig.\ref{ExpApp:fig:muon_chamber2}.(ref.\cite{Chatrchyan:2008aa},\cite{Chatrchyan:2013sba},\cite{Sirunyan:2018fpa})
		
			\begin{figure}[H]
			\centering{}
		    	\includegraphics[width=0.85\textwidth]{Figures/ExpApparatus/muon_chamber2.png}\\
			\caption{Muon detectors structure schema-2\cite{Chatrchyan:2008aa}}
			\label{ExpApp:fig:muon_chamber2}
			\end{figure}
			\FloatBarrier

			%\cite{Chatrchyan:2008aa} %overview 2008
			%\cite{Chatrchyan:2013sba} % 7tev muon detector
			%\cite{Sirunyan:2018fpa} %reco 13tev




