% !TEX root = main.tex

% TODO:
% REMARK:

\section{CP Asymmetry in Top quarks pair and corresponding observables/models}
\label{sec:AcpModelObs}

	With the real 2016 data from CMS, a complete analysis on $t\bar{t}$ CPV is done and mentioned in previous chapters. The analysis focuses on the CPV probable coming from the $t\bar{t}$ in CEDM model and it uses the triple products to be the CP observable to do the measurement on the data. However, there are not only triple products' type observable could be used, but also another type which is characterized as T-even observables; There are also not only CEDM model could be the source of top quarks' CPV, but 2HDM model which has been discussed since decades. 

	\subsection{Models}
	\label{ssec:AcpModel}

		In reference ***0006032, it is particularly organized the information about CPV in top quarks. The usual discussions about CPV in top quarks are related to CEDM model and 2HDM model.

		\subsubsection{ CEDM model}
		\label{sssec:AcpModel_CEDM}

		\subsubsection{ 2HDM model}
		\label{sssec:AcpModel_2HDM}

			The only detected neural Higgs-boson in standard model is CP and flavor conserving automatically. However, the extended-SM Higgs sector could be allowed CP violation. The beyond standard model's 2HDM model is the probable case and one of simple case of Higgs sector with CP violation. The CP violation in the Higgs sector can be evoked in the models where the coupling in the Higgs potential is complex. The complex coupling may cause the CP-violating interaction directly, and the complex VEV's of of Higgs field that could contain CPV effect. Also, the real potential could lead to a ground state in which CP breaks spontaneously.

	\subsection{Observables}
	\label{ssec:AcpObs}

		\subsubsection{Classification}
		\label{sssec:AcpObs_class}

		\subsubsection{Observable property}
		\label{sssec:AcpObs_property}


\FloatBarrier
